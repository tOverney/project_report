\documentclass{article}
\usepackage{graphicx}

\begin{document}

\title{Benchmarking Fermi Microarchitecture}
\author{Tristan Overney, Cl\'{e}ment Humbert}
\date{\today}
\maketitle

\section{Goals}
	The goals of this research is to expose the microarchitecture implemented by
	Nvidia Fermi cards such as: pipeline length, instructions latency, scheduling
	patterns.

\section{Methods}
	To achieve the aforementionned goals, a serie of specially crafted CUDA kernels
	were used. These usually contain large batches of dependent instructions that were 
	timed with the assistance of the clock64() function offered by the CUDA API.

	The benchmark programs have been ran on a machine equipped with a: 
	Nvidia GeForce GTX 580.

\section{Integers multiplication benchmarking}
	This section contains the results obtained through the previously described
	methods using large batches of integer multiplications.

	\subsection{Integer multiplication: threads starting times}
	\includegraphics[width=\linewidth]{"starting_times"}
	\pagebreak

	\subsection{Integer multiplication: threads ending times against thread ids}
	\includegraphics[width=\linewidth]{"ending_times"}
	\pagebreak

	\subsection{Threads total running time against thread ids}
	\includegraphics[width=\linewidth]{"running_times"}
	\pagebreak

	\subsection{Threads running times divided by number of multplications}
	\includegraphics[width=\linewidth]{"latency_estimate"}
	\pagebreak

\section{Interpretation}
\end{document}
